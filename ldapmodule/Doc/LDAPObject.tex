
% $Id: LDAPObject.tex,v 1.1.1.1 2000/02/01 05:41:30 leonard Exp $

\subsection{LDAP Objects}

\noindent
LDAP objects are created by the \code{open()} function defined at the top 
level of the module. The connection is automatically unbound and closed 
during garbage collection.

Most methods initiate an asynchronous request to the LDAP server 
and return a message id that can be used later to retrieve the result. 
The methods ending with \code{_s} are the synchronous form and wait for and
return with
the server's result, \textit{\'{a} la} \code{result()}, or with
\code{None} if no data is expected. See the \code{result()} method
for a description of the data structure returned from the server.

\subsubsection{Exceptions from methods}

Unlike the \C\ library, errors are not returned as result codes, but
are instead turned into exceptions, raised as soon an the error condition 
is detected. The exceptions are accompanied by a dictionary containing
extra information. 

This dictionary contains an entry for the key \code{'desc'} 
for an English description of the error class and \code{'info'} which
contains a string containing more information the server may have sent.

If the exception was one of 
\code{NO_SUCH_OBJECT}, 
\code{ALIAS_PROBLEM}, 
\code{INVALID_DNS_SYNTAX}, 
\code{IS_LEAF}, or 
\code{ALIAS_DEREFERENCING_PROBLEM}, 
then \code{'matched'} will be a key for
the name of the lowest entry (object or alias) that was matched and is a 
truncated form of the name provided or aliased dereferenced.

%%============================================================
%% __methods__

\subsubsection{Methods on LDAP Objects}

\renewcommand{\indexsubitem}{(LDAP method)}

%%------------------------------------------------------------
%% abandon

\begin{funcdesc}{abandon}{msgid}
Abandons or cancels an LDAP operation in progress. The \var{msgid}
should be the message id of an outstanding LDAP operation as returned by
the asynchronous methods \code{search()}, \code{modify()} etc. 
The caller can expect that the result
of an abandoned operation will not be returned from a future call to 
\code{result()}.
\end{funcdesc}

%%------------------------------------------------------------
%% add

\begin{funcdesc}{add}{dn\, modlist}
     \funcline{add_s}{dn\, modlist}
This function is similar to \code{modify()}, except that no operation
integer need be included in the tuples.
\end{funcdesc}

%%------------------------------------------------------------
%% bind

\begin{funcdesc}{bind}{who\, cred\, method}
     \funcline{bind_s}{who\, cred\, method}
     \funcline{simple_bind}{who\, passwd}
     \funcline{simple_bind_s}{who\, passwd}
     \funcline{kerberos_bind_s}{who}
     \funcline{kerberos_bind1}{who}
     \funcline{kerberos_bind1_s}{who}
     \funcline{kerberos_bind2}{who}
     \funcline{kerberos_bind2_s}{who}
After an LDAP object is created, and before any other operations can be
attempted over the connection, a bind operation must be performed.

This method attempts to bind with the LDAP server using 
either simple authentication, or kerberos. The general method \code{bind()}
takes a third parameter, \var{method} which can be one of
\code{AUTH_SIMPLE}, \code{AUTH_KRBV41} or \code{AUTH_KRBV42}.
The \var{cred} parameter is ignored for Kerberos authentication.

Kerberos authentication is only available if the LDAP library and 
the ldap module were compiled with \code{-DWITH_KERBEROS}.
\end{funcdesc}

%%------------------------------------------------------------
%% compare

\begin{funcdesc}{compare}{dn\, attr\, value}
     \funcline{compare_s}{dn\, attr\, value}
Perform an LDAP comparison between the attribute named \var{attr} of 
entry \var{dn}, and the value \var{value}. The synchronous form
returns \code{0} for false, or \code{1} for true.
The asynchronous form returns the message id of the initiates request, 
and the result of the asynchronous compare can be obtained using 
\code{result()}.  

Note that this latter technique yields the answer
by raising the exception objects \code{COMPARE_TRUE} or \code{COMPARE_FALSE}.

A design bug in the library prevents \var{value} from containing nul characters.
\end{funcdesc}

%%------------------------------------------------------------
%% delete

\begin{funcdesc}{delete}{dn}
     \funcline{delete_s}{dn}
Performs an LDAP delete operation on \var{dn}. The asynchronous form
returns the message id of the initiated request, and the result can be obtained
from a subsequent call to \code{result()}.
\end{funcdesc}

%%------------------------------------------------------------
%% destroy_cache

\begin{funcdesc}{destroy_cache}{}
Turns off caching and removed it from memory.
\end{funcdesc}

%%------------------------------------------------------------
%% disable_cache

\begin{funcdesc}{disable_cache}{}
Temporarily disables use of the cache. New requests are not cached, and
the cache is not checked when returning results. Cache contents are not
deleted.
\end{funcdesc}

%%------------------------------------------------------------
%% enable_cache

\begin{funcdesc}{enable_cache}{\optional{timeout=\code{NO_LIMIT}\, 
                               \optional{maxmem=\code{NO_LIMIT}}}}
Using a cache often greatly improves performance. By default the cache
is disabled. Specifying \var{timeout} in seconds is used to decide how long
to keep cached requests. The \var{maxmem} value is in bytes, and is used
to set an upper bound on how much memory the cache will use. A value of
\code{NO_LIMIT} for either indicates unlimited. 
Subsequent calls to
\code{enable_cache} can be used to adjust these parameters.

This and other caching methods are not available if the library and the 
ldap module were compiled with \code{-DNO_CACHE}.
\end{funcdesc}

%%------------------------------------------------------------
%% flush_cache

\begin{funcdesc}{flush_cache}{}
Deletes the cache's contents, but does not affect it in any other way.
\end{funcdesc}

%%------------------------------------------------------------
%% modify

\begin{funcdesc}{modify}{ dn\, modlist }
     \funcline{modify_s}{ dn\, modlist }
Performs an LDAP modify operation on an entry's attributes. 
\var{dn} is the DN of the entry to modify,
and \var{modlist} is the list of modifications to make to the entry.

Each element of the list \var{modlist} should be a tuple of the form 
\code{(mod_op,mod_type,mod_vals)},
where \var{mod_op} is the operation (one of \code{MOD_ADD}, \code{MOD_DELETE},
or \code{MOD_REPLACE}),
\var{mod_type} is a string indicating the attribute type name, and 
\var{mod_vals} is either a string value or a list of string values to add, 
delete or replace respectively.  For the delete operation, \var{mod_vals}
may be \code{None} indicating that all attributes are to be deleted.

The asynchronous \code{modify()} returns the message id of the 
initiated request.

See the \code{ber_*} methods for decoding Basic Encoded ASN.1 values.
(To be implemented.)
\end{funcdesc}

%%------------------------------------------------------------
%% modrdn

\begin{funcdesc}{modrdn}{dn\, newrdn \optional{\, delold=\code{1}}}
     \funcline{modrdn_s}{dn\, newrdn \optional{\, delold=\code{1}}}
Perform a modify RDN operation. These routines take \var{dn}, the DN
of the entry whose RDN is to be changed, and \var{newrdn}, the new RDN to
give to the entry. The optional parameter \var{delold} is used to specify
whether the old RDN should be kept as an attribute of the entry or not.
The asynchronous version returns the initiated message id.

This actually corresponds to the \code{modrdn2*} routines in the \C\ library.
\end{funcdesc}

%%------------------------------------------------------------
%% result

\begin{funcdesc}{result}{\optional{ msgid=\code{RES_ANY} \optional{\, all=\code{1} \optional{\, timeout=\code{-1}}}}}
This method is used to wait for and return the result of an operation
previously initiated by one of the LDAP asynchronous operation routines
(eg \code{search()}, \code{modify()}, etc.) They all returned
an invocation identifier (a message id) upon successful initiation of
their operation. This id is guaranteed to be unique across an LDAP session,
and can be used to request the result of a specific operation via the
\var{msgid} parameter of the \code{result()} method.

If the result of a specific operation is required, \var{msgid} should
be set to the invocation message id returned when the operation was
initiated; otherwise \code{RES_ANY} should be supplied. 

The \var{all}
parameter only has meaning for \code{search()} responses and is used to select 
whether a single entry of the search response should be returned, or to
wait for \emph{all} the results of the search before returning. 
%See \code{search()} for more details.

A search response is made up of zero or more search entries followed by
a search result. If \var{all} is \code{0}, search entries will be returned one
at a time as they come in, via separate calls to \code{result()}. If
\var{all} is \code{1}, the search response will be returned in its 
entirety, ie after all entries and the final search result have been
received.

The method returns a tuple of the form 
\code{(\var{result_type}, \var{result_data})}.
The \var{result_type} is a string, being one of:
\code{'RES_BIND'}, \code{'RES_SEARCH_ENTRY'}, \code{'RES_SEARCH_RESULT'}, 
\code{'RES_MODIFY'}, \code{'RES_ADD'}, \code{'RES_DELETE'}, 
\code{'RES_MODRDN'}, or \code{'RES_COMPARE'}.

The constants \code{RES_*} are set to these strings, for convenience.

See \code{search()} for a description of the search result's \var{result_data},
otherwise the \var{result_data} is normally meaningless.

The \code{result()} method will block for \var{timeout} seconds, or 
indefinitely if \var{timeout} is negative. 
A timeout of \code{0} will effect a poll. 
The timeout can be expressed as a floating-point value.

If a timeout occurs, the tuple \code{(None,None)} is returned.
% XXX - should this raise a TIMEOUT exception?

\end{funcdesc}

%%------------------------------------------------------------
%% search

\begin{funcdesc}{search}{base\, scope\, filter\optional{\, 
	attrlist=\code{None}\optional{\, attrsonly=\code{0}}}}
     \funcline{search_s}{base\, scope\, filter\optional{\, 
     	attrlist=\code{None}\optional{\, attrsonly=\code{0}}}}
    \funcline{search_st}{base\, scope\, filter\optional{\,
    	attrlist=\code{None}\optional{\, attrsonly=\code{0} \optional{\, 
		timeout=\code{-1}}}}}
Perform an LDAP search operation, with \var{base} as the DN of the entry
at which to start the search, \var{scope} being one of 
\code{SCOPE_BASE} (to search the object itself), 
\code{SCOPE_ONELEVEL} (to search the object's immediate children), or
\code{SCOPE_SUBTREE} (to search the object and all its descendants).

\var{filter} is a string representation of the filter to apply in
the search. Simple filters can be specified as
\code{"\var{attribute_type}=\var{attribute_value}"}. 
More complex filters are specified using a prefix notation according to 
the following BNF:
\begin{eqnarray*}
	\var{filter}	&::=& \code{"("} \ \var{filtercomp} \ \code{")"}
\\	\var{filtercomp}&::=& \var{and} \ |\ \var{or}\ |\ \var{not} 
			      \ |\ \var{simple}
\\	\var{and}	&::=& \code{"\&"} \ \var{filterlist}
\\	\var{or}	&::=& \code{"|"} \ \var{filterlist}
\\	\var{not}	&::=& \code{"!"} \ \var{filter}
\\	\var{filterlist}&::=& \var{filter} \ |\ \var{filter} \ \var{filterlist}
\\	\var{simple}	&::=& \var{attributetype}\ \var{filtertype} 
			      \ \var{attributevalue}
\\	\var{filtertype}&::=& \code{"="} \ |\ \code{"\~="}\ |\ \code{"<="}
		              \ |\ \code{">="}
\end{eqnarray*}

When using the asynchronous form and \code{result()}, the \var{all}
parameter affects how results come in.
For \var{all} set to \code{0}, 
result tuples trickle in (with the same message id), and with the result
type \code{RES_SEARCH_ENTRY}, until the final result which has 
a result type of \code{RES_SEARCH_RESULT} and a (usually) empty data field.
When \var{all} is set to \code{1}, only one result is returned, with a
result type of \code{RES_SEARCH_RESULT}, and all the result tuples listed 
in the data field.

Each result tuple is of the form \code{(\var{dn},\var{attrs})}, 
where \var{dn} is a string containing
the DN (distinguished name) of the entry, and 
\var{attrs} is a dictionary
containing the attributes associated with the entry. 
The keys of \var{attrs} are strings, 
and the associated values are lists of strings.

The DN in \var{dn} is extracted using the underlying \code{ldap_get_dn()},
which may raise an exception if the DN is malformed.

If \var{attrsonly} is non-zero, the values of \var{attrs} will be meaningless
(they are not transmitted in the result).

The retrieved attributes can be limited with the \var{attrlist} parameter.
If \var{attrlist} is \code{None}, all the attributes of each entry are returned.

The synchronous form with timeout, \code{search_st()}, will block for at most
\var{timeout} seconds (or indefinitely if \var{timeout} is negative). A
\code{TIMEOUT} exception is raised if no result is received within the
time.
\end{funcdesc}

%%------------------------------------------------------------
%% set_cache_options

\begin{funcdesc}{set_cache_options}{option}
Changes the caching behaviour. Currently supported options are
    \code{CACHE_OPT_CACHENOERRS}, which suppresses caching of requests
    	that resulted in an error, and
    \code{CACHE_OPT_CACHEALLERRS}, which enables caching of all requests.
The default behaviour is not to cache requests that result in errors, except 
those that result in a \code{SIZELIMIT_EXCEEDED} exception.
\end{funcdesc}

%%------------------------------------------------------------
%% set_rebind_proc

\begin{funcdesc}{set_rebind_proc}{func}
If a referral is returned from the server, automatic
re-binding can be achieved by providing a function that accepts as an argument
the newly opened LDAP object and returns the tuple \code{(who, cred, method)}.

Passing a value of \code{None} for \var{func} will disable
this facility. 

Because of restrictions in the implementation, only one
rebinding function is supported at any one time. This method is only
available if the module and library were compiled with \code{-DLDAP_REFERRALS}.
\end{funcdesc}

%%------------------------------------------------------------
%% ufn_setfilter

\begin{funcdesc}{ufn_setfilter}{filtername}
       \funcline{ufn_setprefix}{prefix}
       \funcline{ufn_search_s}{url\optional{\, attrsonly=\code{0}}}
       \funcline{ufn_search_st}{url\optional{\, attrsonly=\code{0} \optional{\, timeout=\code{-1}}}}
See the LDAP library manual pages for more information on these
`user-friendly name' functions.
\end{funcdesc}

%%------------------------------------------------------------
%% unbind

\begin{funcdesc}{unbind_s}{}
       \funcline{unbind}{}
This call is used to unbind from the directory, terminate the current
association, and free resources. Once called, the connection to the
LDAP server is closed and the LDAP object is invalid. Further invocation
of methods on the object will yield an exception.

The \code{unbind} and \code{unbind_s} methods are identical, and are 
synchronous in nature
\end{funcdesc}

%%------------------------------------------------------------
%% uncache_entry

\begin{funcdesc}{uncache_entry}{dn}
Removes all cached entries that make reference to \var{dn}. This should be
used, for example, after doing a \code{modify()} involving \var{dn}.
\end{funcdesc}

%%------------------------------------------------------------
%% uncache_request

\begin{funcdesc}{uncache_request}{msgid}
Remove the request indicated by \var{msgid} from the cache.
\end{funcdesc}

%%------------------------------------------------------------
%% url_search

\begin{funcdesc}{url_search_s}{url\, \optional{attrsonly=\code{0}}}
       \funcline{url_search_s}{url\, \optional{attrsonly=\code{0} \optional{\, timeout=\code{-1}}}}
These routine works much like \code{search_s*}, except that many search
parameters are pulled out of the URL \var{url}. 

LDAP URLs look like this: 

\code{"ldap://\var{host}\optional{:\var{port}}/\var{dn}\optional{?\var{attributes}\optional{?\var{scope}\optional{?\var{filter}}}}"}

where \var{scope} is one of \code{base} (default), \code{one} or \code{sub},
and \var{attributes} is a comma-separated list of attributes to be retrieved.

URLs wrapped in angle-brackets and/or preceded by \code{"URL:"} are 
tolerated.
\end{funcdesc}

%%============================================================
%% attributes

\subsubsection{Attributes on LDAP Objects}

Each LDAP object also sports the following attributes.

%%------------------------------------------------------------
%% deref

\begin{datadesc}{deref}
    Controls for when an automatic dereference of a referral occurs.
    This must be one of
    \code{DEREF_NEVER}, \code{DEREF_SEARCHING}, \code{DEREF_FINDING},
    or \code{DEREF_ALWAYS}.
\end{datadesc}

%%------------------------------------------------------------
%% errno

\begin{datadesc}{errno}
   \dataline{error}
   \dataline{matched}
    These read-only attributes are set after an exception has been raised, and
    are also included with the value raised. See the section
    `Exceptions from methods', above.
\end{datadesc}

%%------------------------------------------------------------
%% lberoptions

\begin{datadesc}{lberoptions}
    Options for the BER library.
\end{datadesc}

%%------------------------------------------------------------
%% options

\begin{datadesc}{options}
    General options. This field is the bitiwse OR of the flags
	\code{OPT_REFERRALS} (follow referrals), and
	\code{OPT_RESTART}   (restart the \var{select} system call
			      when interrupted).
\end{datadesc}

%%------------------------------------------------------------
%% refhoplimit

\begin{datadesc}{refhoplimit}
    Maximum number of referrals to follow before raising an exception.
    Defaults to 5.
\end{datadesc}

%%------------------------------------------------------------
%% sizelimit

\begin{datadesc}{sizelimit}
    Limit on size of message to receive from server. 
    Defaults to \code{NO_LIMIT}.
\end{datadesc}

%%------------------------------------------------------------
%% timelimit

\begin{datadesc}{timelimit}
    Limit on waiting for any response. 
    Defaults to \code{NO_LIMIT}.
\end{datadesc}

%%------------------------------------------------------------
%% valid

\begin{datadesc}{valid}
    If zero, the connection has been unbound. See \code{unbind()} for
    more information.
\end{datadesc}

